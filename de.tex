\documentclass[]{article}
\usepackage[a4paper, total={7in, 11in}]{geometry}

\usepackage{fontawesome5}
\usepackage{hyperref}
\usepackage{xcolor}
\usepackage{enumitem}
\usepackage[ngerman]{babel}

\hypersetup{hidelinks}
\pagenumbering{gobble}

\usepackage{ulem}
\usepackage[]{titlesec}
    \titleformat{\section}
    {\scshape\Large}
    {}
    {0em}
    {}[{\titlerule}]

\begin{document}

\begin{center}
    \textbf{\Large @hiegz} \\[3pt]
    \textbf{\huge Volodymyr But} \\[7pt]

    \faIcon{globe} \href{https://www.hiegz.com}{hiegz.com}
    \hspace{5pt}
    \faIcon{inbox} hello@hiegz.com \\[2pt]

    \faIcon{github} \href{https://github.com/hiegz}{github.com/hiegz}
    \hspace{5pt}
    \faIcon{linkedin} \href{https://linkedin.com/in/hiegz}{in/hiegz} \\[10pt]

    Web-Entwickler und professioneller Tee-Degustator. Details folgen unten.
\end{center}

\section{Ausbildung}

\begin{itemize}[leftmargin=0.15in, rightmargin=0.15in, label={}]
    \item {\large\bfseries Hochschule Trier} \hfill 09/24 - heute \\
        {\itshape Bachelor of Science in K"unstlicher Intelligenz und Maschinellem Lernen \\ Fachbereich Informatik}

        \begin{itemize}
            \item Aktueller Notendurchschnitt: 1.7
        \end{itemize}
\end{itemize}

\section{Projekte}

\begin{itemize}[leftmargin=0.15in, label={}]
    \item
        \href{https://github.com/hiegz/chester}{\large\bfseries Chester} \\
        {\itshape Schachprogramm mit UCI-Schnittstelle}

        Bis heute implementiert:

        \begin{itemize}
            \item Bitboard-basierte Brettdarstellung
            \item effiziente Generierung von pseudo-legalen Zügen unter Verwendung vorab berechneter Angriffstabellen und Magic Bitboards
        \end{itemize}

        in Entwicklung:

        \begin{itemize}
            \item Zugvalidierung erforderlich für die Erzeugung legaler Züge
            \item Stellenbewertung
            \item Universal Chess Interface (UCI) Protokoll
        \end{itemize}

    \item
        \href{https://github.com/hiegz/fuizon}{\large\bfseries Fuizon} \\
        {\itshape Plattformübergreifende TUI-Bibliothek für Zig}

        Bis heute implementiert:

        \begin{itemize}
            \item niedrigstufige Abstraktionsschicht für Terminal I/O  (alternativer Bildschirm, Rohmodus, Polling, Tastaturereignisse, usw.)
            \item effiziente Rendering-Engine, bei der jeder Render-Zyklus nur die Unterschiede zwischen vorherigem und aktuellem Frame zeichnet.
            \item Inkrementeller Constraint-Solver unter Verwendung des Cassowary-Algorithmus für dynamische Layout-Modelle
            \item minimaler Widget-Bausatz
        \end{itemize}

        in Entwicklung:

        \begin{itemize}
            \item Unterstützung für Mausereignisse
            \item Umfangreicheres Layout- und Widget-Set
            \item Dokument-Objektmodell (DOM)
        \end{itemize}
\end{itemize}

\section{Sprachen}

\hspace{0.15in}
\begin{tabular}{@{}ll}
    \textbf{Englisch}   & C1 \\[10pt]
    \textbf{Deutsch}    & C1 \\[10pt]
    \textbf{Ukrainisch} & Muttersprache \\[10pt]
    \textbf{Russisch}   & Muttersprache \\
\end{tabular}

\end{document}
