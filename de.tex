\documentclass[]{article}
\usepackage[a4paper, total={7in, 11in}]{geometry}

\usepackage{fontawesome5}
\usepackage{hyperref}
\usepackage{xcolor}
\usepackage{enumitem}
\usepackage[ngerman]{babel}

\hypersetup{hidelinks}
\pagenumbering{gobble}

\usepackage{ulem}
\usepackage[]{titlesec}
    \titleformat{\section}
    {\scshape\Large}
    {}
    {0em}
    {}[{\titlerule}]

\begin{document}

\begin{center}
    \textbf{\Large @hiegz} \\[3pt]
    \textbf{\huge Volodymyr But} \\[7pt]

    \faIcon{globe} \href{https://www.hiegz.com}{hiegz.com}
    \hspace{5pt}
    \faIcon{inbox} hello@hiegz.com \\[2pt]

    \faIcon{github} \href{https://github.com/hiegz}{github.com/hiegz}
    \hspace{5pt}
    \faIcon{linkedin} \href{https://linkedin.com/in/hiegz}{in/hiegz} \\[10pt]

    Professioneller Tee-Degustator
\end{center}

\section{Ausbildung}

\begin{itemize}[leftmargin=0.15in, rightmargin=0.15in, label={}]
    \item {\large\bfseries Hochschule Trier} \hfill 09/24 - heute \\
        {\itshape Bachelor of Science in K"unstlicher Intelligenz und Maschinellem Lernen \\ Fachbereich Informatik}

        \begin{itemize}
            \item Aktueller Notendurchschnitt: 1.7
        \end{itemize}
\end{itemize}

\section{Berufserfahrung}

\begin{itemize}[leftmargin=0.15in, rightmargin=0.15in, label={}]
    \item {\large\bfseries Open Source}

        \:\: {\faIcon{github} \href{https://github.com/hiegz}{\ttfamily \underline{github.com/hiegz}}}

        \vspace{7pt}
        \hspace{5pt}
        Aktuelle Beiträge:

        \hspace{20pt}
        \begin{tabular}{@{}lp{8cm}}
            \href{https://github.com/hiegz/chester}{\ttfamily hiegz/chester} & UCI Schachengine \\[3pt]
            \href{https://github.com/hiegz/fuizon}{\ttfamily hiegz/fuizon}   & Plattformübergreifende TUI-Bibliothek für Zig \\[3pt]
            \href{https://github.com/hiegz/raspi}{\ttfamily hiegz/raspi}     & Raspberry Pi 4 Model B als \verb|WLAN|-Hotspot mit Internetzugang über einen \verb|VPN|-Server \\[3pt]
            \href{https://github.com/hiegz/aes}{\ttfamily hiegz/aes}         & Advanced Encryption Standard (AES) \\[3pt]
            \href{https://github.com/hiegz/des}{\ttfamily hiegz/des}         & Data Encryption Standard (DES)
        \end{tabular}
\end{itemize}

\section{Kompetenzen}

\vspace{5pt}
\begin{minipage}[t]{0.49\linewidth}
    \begin{itemize}[leftmargin=0.15in, rightmargin=0.15in, label={}]
        \item {\large\bfseries Frontend}

            \begin{itemize}
                \item Zentrale Webtechnologien (\verb|HTML|, \verb|CSS|,
                    \verb|JS|, \verb|React|, und praktisch alles andere)
                \item Entwicklung von Desktop-Anwendungen f"ur \verb|Windows| mit \verb|WPF| oder \verb|UWP|
            \end{itemize}
    \end{itemize}
\end{minipage}
\hspace{10pt}
\begin{minipage}[t]{0.49\linewidth}
    \begin{itemize}[leftmargin=0.15in, rightmargin=0.15in, label={}]
        \item {\large\bfseries Backend}
            \begin{itemize}
                \item \verb|Java|/\verb|Spring Boot|, \verb|Go|, \verb|NodeJS|
                \item \verb|PostgreSQL|, \verb|MySQL|, \verb|MongoDB|
                \item \verb|Amazon|, \verb|Cloudflare| (e.g. \verb|EC2|, \verb|S3|, \verb|R2|, ...)
            \end{itemize}
    \end{itemize}
\end{minipage}
\\[20pt]
\begin{minipage}[t]{0.49\linewidth}
    \begin{itemize}[leftmargin=0.15in, rightmargin=0.15in, label={}]
        \item {\large\bfseries Systemprogrammierung}

            \begin{itemize}
                \item Low-Level-Programmierung in \verb|C/C++|, \verb|Rust|, und \verb|Zig|
                \item Erfahrung mit \verb|Make| und \verb|CMake|
                \item \verb|POSIX|-Systemschnittstelle
                \item plattformübergreifendes Terminal-I/O (siehe \href{https://github.com/hiegz/fuizon}{\ttfamily \underline{hiegz/fuizon}})
                \item Nicht-blockierendes I/O, asynchrone Event-Loops (siehe \href{https://github.com/hiegz/rcs}{\ttfamily \underline{hiegz/rcs}})
                \item Speicherverwaltung (siehe \href{https://github.com/hiegz/heap}{\ttfamily \underline{hiegz/heap}})
                \item Sprachübergreifende Integration über FFI (siehe
                    \href{https://github.com/hiegz/crossterm-ffi}{\ttfamily \underline{hiegz/crossterm-ffi}} und
                    \href{https://github.com/hiegz/fuizon}{\ttfamily \underline{hiegz/fuizon}})
            \end{itemize}
    \end{itemize}
\end{minipage}
\hspace{10pt}
\begin{minipage}[t]{0.49\linewidth}
    \begin{itemize}[leftmargin=0.15in, rightmargin=0.15in, label={}]
        \item {\large\bfseries Systemadministration}

            \begin{itemize}
                \item Erfahrung mit Windows und verschiedenen Linux-Distributionen (Kali, Ubuntu, Mint, Arch, usw.)
            \end{itemize}

            \hspace{5pt}
            Zum Beispiel, betrachten Sie meine Einrichtung

            \begin{itemize}
                \item Primäre Entwicklungsumgebung unter Arch Linux, angepasst für den täglichen Einsatz und systemnahe Effizienz
                    (siehe \href{https://github.com/hiegz/dotfiles}{\ttfamily \underline{hiegz/dotfiles}})
                \item Sekundäre Umgebung unter Windows für plattformübergreifende Tests und Softwarekompatibilität
                \item Raspberry Pi als Dualband-\verb|WLAN|-Access-Point eingerichtet,
                    der sämtlichen Datenverkehr über einen entfernten \verb|VPN|-Server tunnelt
                    (siehe \href{https://github.com/hiegz/raspi}{\ttfamily \underline{hiegz/raspi}})
            \end{itemize}
    \end{itemize}
\end{minipage}

\vspace{5pt}
\section{Sprachen}

\hspace{0.15in}
\textbf{Englisch} \: C1 \hspace{20pt} \textbf{Deutsch} \: C1 \hspace{20pt} \textbf{Ukrainisch} \: Muttersprache \hspace{20pt} \textbf{Russisch} \: Muttersprache

\end{document}
